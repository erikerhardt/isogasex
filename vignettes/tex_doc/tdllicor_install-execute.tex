\renewcommand{\ADAfilename}{tdl_inst} % for R plot filenames
\chapter{Install and execute}
\label{ch:tdllicor_install_execute}



%%%%%%%%%%%%%%%%%%%%%%%%%%%%%%%%%%%%%%%%%%%%%%%%%%%%%%%%%%%%%%%%%%%%%%%%%%%%%%%%
\section{Install software}

\R\ and Perl are required.  Rstudio may make your experience more enjoyable.
\begin{enumerate}
\item \R\ (requires R 2.15.1+)
  \begin{enumerate}
  \item Windows (\url{http://cran.r-project.org/bin/windows/base/}) or
  \item Mac (\url{http://cran.r-project.org/bin/macosx/})
  \end{enumerate}
\item Rstudio
  \begin{enumerate}
  \item Rstudio (\url{http://www.rstudio.org/download/}) nicer GUI
  \end{enumerate}
\item Perl
  \begin{enumerate}
  \item Mac and linux, you already have it installed
  \item Windows
    \begin{enumerate}
    \item \url{http://www.activestate.com/activeperl/downloads}
    \item Change the path variable to include \verb|C:\Perl\bin| (or whatever it is on your machine, and use backslashes): \url{http://www.java.com/en/download/help/path.xml}
    \end{enumerate}
  \end{enumerate}
\end{enumerate}


%%%%%%%%%%%%%%%%%%%%%%%%%%%%%%%%%%%%%%%%%%%%%%%%%%%%%%%%%%%%%%%%%%%%%%%%%%%%%%%%
\section{Install the \tdllicor\ package}

Install the tdllicor package
From the shared dropbox folder \verb|./TDL_R_scripts/| install the current version of tdllicor.
The package installation file will appear in a folder named something like
\begin{itemize}
\item \verb|./tdllicor_0.1-21_20121017-R2.15.1-template4/|
\end{itemize}
and will include two files, such as
\begin{itemize}
\item the package: \lst{tdllicor_0.1-21.tar.gz}
\item and the template: \lst{tdllicor_template4.xls}
\end{itemize}

There doesn't seem to be a way to install local packages within R anymore, so it has to be done from the command prompt.
From your shell or command prompt:
\begin{enumerate}
\item Windows, open your command prompt (cmd), change to directory with install file, and install it (you need to get the correct path, filename, and R command name).
  \begin{enumerate}
  \item {\footnotesize \verb|cd C:\[your path]\TDL_R_scripts\tdllicor_0.1-21_20121017-R2.15.1-template4| }
  \item {\footnotesize \verb|"C:\Program Files\R\R-2.15.1\bin\x64\R" CMD INSTALL tdllicor_0.1-21.tar.gz| }
  \end{enumerate}
\item Mac, I can't walk you through this since I don't have access to a Mac right now, but it's similar, probably something like this:
  \begin{enumerate}
  \item {\footnotesize \verb|cd ~/[your path]/TDL_R_scripts/tdllicor_0.1-21_20121017-R2.15.1-template4| }
  \item {\footnotesize \verb|R CMD INSTALL tdllicor_0.1-21.tar.gz| }
  \end{enumerate}
\end{enumerate}

Copy the template file to your data directory before you rename and edit it.


%%%%%%%%%%%%%%%%%%%%%%%%%%%%%%%%%%%%%%%%%%%%%%%%%%%%%%%%%%%%%%%%%%%%%%%%%%%%%%%%
\section{Analyze your TDL and/or Licor data with \tdllicor}

{\em Try to avoid using spaces in your directory names (folder names) and file names.} \\

\noindent
Analyze your TDL and/or Licor data
\begin{enumerate}
\item Create a new data analysis directory, such as \verb|./Name_WT001|
\item Copy your TDL and Licor files into your new dir
\item Copy the \verb|tdllicor_template4.xls| into your new dir
\item Append a meaningful suffix to the \verb|.xls| file, such as \verb|tdllicor_template4_Name_WT001.xls|
\item Edit the \verb|.xls| file and specify the TDL and Licor filenames, and make any other necessary changes to template inputs.
\item Load the library: (see end of doc for the remaining steps all together)
  \begin{enumerate}
  \item \lst{library(tdllicor)}
  \end{enumerate}
\item Assign the xls filename to a variable
  \begin{enumerate}
  \item \lst{input.fn <- "tdllicor_template4_Name_WT001.xls"}
  \end{enumerate}
\item Specify your new dir (using forward slashes for all systems is preferred)
  \begin{enumerate}
  \item Mac, Linux:
    \begin{enumerate}
    \item \lst{path <- "/Users/username/Analysis/Name_WT001"}
    \end{enumerate}
  \item Windows:
    \begin{enumerate}
    \item \lst{path <- "C:/Dropbox/Analysis/Name_WT001"}
    \end{enumerate}
  \end{enumerate}
\item Run the analysis
  \begin{enumerate}
  \item a.  \lst{tdllicor(input.fn, path)}
  \end{enumerate}
\item It will create an \verb|./out| directory where output files are created.
\item Documentation is available with \lst{?tdllicor}
\end{enumerate}

Use a block of code like this, set your file name and directory, and copy/paste into R.
\begin{Hchunk}
\begin{footnotesize}
\begin{Hinput}
\ttfamily\noindent
\hlprompt{{\ }}\hlfunctioncall{library}\hlkeyword{(}\hlsymbol{tdllicor}\hlkeyword{)}\mbox{}
\normalfont
\end{Hinput}


\begin{Hinput}
\ttfamily\noindent
\hlprompt{{\ }}\hlcomment{\usebox{\hlfootnotesizeboxhash}{\ }?tdllicor}\hspace*{\fill}\\
\hlstd{}\hlprompt{{\ }}\hlsymbol{input.fn}{\ }\hlassignement{\usebox{\hlfootnotesizeboxlessthan}-}{\ }\hlstring{"tdllicor\usebox{\hlfootnotesizeboxunderscore}template4\usebox{\hlfootnotesizeboxunderscore}Name\usebox{\hlfootnotesizeboxunderscore}WT001.xls"}\mbox{}
\normalfont
\end{Hinput}


\begin{Hinput}
\ttfamily\noindent
\hlprompt{{\ }}\hlsymbol{path}{\ }\hlassignement{\usebox{\hlfootnotesizeboxlessthan}-}{\ }\hlstring{"/Users/username/Analysis/Name\usebox{\hlfootnotesizeboxunderscore}WT001"}{\ }\hlcomment{\usebox{\hlfootnotesizeboxhash}{\ }Mac}\mbox{}
\normalfont
\end{Hinput}


\begin{Hinput}
\ttfamily\noindent
\hlprompt{{\ }}\hlsymbol{path}{\ }\hlassignement{\usebox{\hlfootnotesizeboxlessthan}-}{\ }\hlstring{"C:/Dropbox/Analysis/Name\usebox{\hlfootnotesizeboxunderscore}WT001"}{\ }{\ }{\ }{\ }{\ }{\ }\hlcomment{\usebox{\hlfootnotesizeboxhash}{\ }Windows}\mbox{}
\normalfont
\end{Hinput}


\begin{Hinput}
\ttfamily\noindent
\hlprompt{{\ }}\hlfunctioncall{tdllicor}\hlkeyword{(}\hlsymbol{input.fn}\hlkeyword{,}{\ }\hlsymbol{path}\hlkeyword{)}\mbox{}
\normalfont
\end{Hinput}


\end{footnotesize}
\end{Hchunk}


% END

