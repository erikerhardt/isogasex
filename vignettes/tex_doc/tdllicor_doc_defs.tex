%  ADA1-defs.tex
%  command definitions

% begin verbatim environments with good line spacing
\newcommand{\bVerb}        {\renewcommand{\baselinestretch}{.66} \small        \begin{Verbatim}[samepage=true]} % use to keep a block of verbatim on the same page
\newcommand{\bverb}        {\renewcommand{\baselinestretch}{.66} \small        \begin{verbatim}}
\newcommand{\everb}        {\renewcommand{\baselinestretch}{1} \normalsize}
\newcommand{\bVerbsmall}   {\renewcommand{\baselinestretch}{.66} \small        \begin{Verbatim}[samepage=true]} % use to keep a block of verbatim on the same page
\newcommand{\bverbsmall}   {\renewcommand{\baselinestretch}{.66} \small        \begin{verbatim}}
\newcommand{\bVerbfootnote}{\renewcommand{\baselinestretch}{.66} \footnotesize \begin{Verbatim}[samepage=true]} % use to keep a block of verbatim on the same page
\newcommand{\bverbfootnote}{\renewcommand{\baselinestretch}{.66} \footnotesize \begin{verbatim}}
\newcommand{\bVerbscript}  {\renewcommand{\baselinestretch}{.66} \scriptsize   \begin{Verbatim}[samepage=true]} % use to keep a block of verbatim on the same page
\newcommand{\bverbscript}  {\renewcommand{\baselinestretch}{.66} \scriptsize   \begin{verbatim}}
\newcommand{\bVerbtiny}    {\renewcommand{\baselinestretch}{.66} \tiny         \begin{Verbatim}[samepage=true]} % use to keep a block of verbatim on the same page
\newcommand{\bverbtiny}    {\renewcommand{\baselinestretch}{.66} \tiny         \begin{verbatim}}
%% USE: Verbatim that can go between pages
% \bverb
%   text
% \end{verbatim}
% \everb
%% OR, Verbatim that should all be on one page
% \bVerb
%   text
% \end{Verbatim}
% \everb

%% Program listings
% \noindent
% Program editor contents:
% \begin{lstlisting}[language=SAS]
% \end{lstlisting}

% \noindent
% Output window contents:


\newcommand{\R}{R} % R programming language

% \lst{} to list inline code (helps with underscore program names)
\newcommand{\lst}[1]{\lstinline[language=R,basicstyle=\small\ttfamily]{#1}} %\normalsize
% Note: \Sexpr{signif(sqrt(9),5)}  will be replaced by the string '3' (without any quotes).

\newcommand{\ADAfilename}{ADA1_notes}
\newcommand{\FigPlotHt}{3in} %

% for plotting figures from R
\newcommand{\printplot}[2]{% #1 = <<label>> name
\begin{center}
\includegraphics[keepaspectratio=true,height=#2]{images/\ADAfilename -#1.pdf}  % #2 = \FigurePlotHeight
\end{center}
}

\newcommand{\printplotTwo}[2]{% #1 = <<label>> name
\begin{center}
\includegraphics[keepaspectratio=true,width=0.495\textwidth]{images/\ADAfilename -#1.pdf}  % #2 = \FigurePlotHeight
\includegraphics[keepaspectratio=true,width=0.495\textwidth]{images/\ADAfilename -#2.pdf}  % #2 = \FigurePlotHeight
\end{center}
}

\newcommand{\printplotTwosmaller}[2]{% #1 = <<label>> name
\begin{center}
\includegraphics[keepaspectratio=true,width=0.43\textwidth]{images/\ADAfilename -#1.pdf}  % #2 = \FigurePlotHeight
\includegraphics[keepaspectratio=true,width=0.43\textwidth]{images/\ADAfilename -#2.pdf}  % #2 = \FigurePlotHeight
\end{center}
}


\newcommand{\Rcodeintro}{\noindent{\small\textsf{\# R code \#}}\vspace{-1ex}}


\newcommand{\xClickerQ}[1]{%
\begin{mdframed}[
    linecolor=Black!25
  , backgroundcolor=White %Black!00
  , innerleftmargin=3ex
  , innerrightmargin=3ex
  , skipabove=3ex
  , skipbelow=3ex
  , nobreak=true]
\rule{1ex}{1ex}\hfill {\large {\sc Clicker}{\em Q}\,s \quad --- \quad #1} \hfill\rule{1ex}{1ex}
\end{mdframed}
}%


