%  ADA1_envi.tex
%  Package environment

% comment this on final compilation
%\usepackage{showkeys}  % shows symbolic references in the margin, [notcite] to not show citations

\usepackage[twoside, bindingoffset=0.5in, includeheadfoot, hmargin=0.5in, vmargin=0.5in]{geometry}

\usepackage{fancyhdr}
\pagestyle{fancy}
% with this we ensure that the chapter and section
% headings are in lowercase.
\renewcommand{\chaptermark}[1]{%
  \markboth{#1}{}}
\renewcommand{\sectionmark}[1]{%
  \markright{\thesection\ #1}}
%\renewcommand{\subsectionmark}[1]{%
%  \markleft{\thesubsection\ #1}}
\fancyhf{} % delete current header and footer
\fancyhead[LE,RO]{\thepage}  %\bfseries
\fancyhead[LO]{\rightmark}
\fancyhead[RE]{\leftmark}
\renewcommand{\headrulewidth}{0.5pt}
\renewcommand{\footrulewidth}{0pt}
\addtolength{\headheight}{0.5pt} % space for the rule
\fancypagestyle{plain}{%
  \fancyhead{} % get rid of headers on plain pages
  \renewcommand{\headrulewidth}{0pt} % and the line
}

\usepackage{color}
\usepackage[dvipsnames]{xcolor}

\usepackage{mdframed} % gray frames around code  \begin{mdframed} \end{mdframed}
\mdfsetup{%
    linecolor=Black!10
  , backgroundcolor=Black!05
  , innerleftmargin=0pt
  , innerrightmargin=0pt
  , skipabove=1ex
  , skipbelow=1ex
  , splittopskip=2ex
  , splitbottomskip=1ex
  , nobreak=false
  , needspace=1in
}

\usepackage{nameref} % to reference chapter names in RFunctions chapter
\usepackage{url} % to reference urls

\usepackage{graphicx}
%\usepackage{subfigure}
\DeclareGraphicsExtensions{.eps,.ps,.pdf,.jpg}

\usepackage{amsmath}    % math enhancements latex2e (replaces amstex)
\usepackage{amsthm}     % proclaims theoremstyle/proof environments
%\usepackage{amscd}      % commutative Diagram environment
\usepackage{amssymb}    % AMSFonts and symbols
%\usepackage{eucal}      % Euler Cal/Script Fonts
%\usepackage{latexsym}   % latex symbols (like \pounds)
\usepackage{verbatim}   % for verbatim environment
\usepackage{fancyvrb}   % for fancy verbatim environment, e.g., to keep blocks on the same page http://www.ctex.org/documents/packages/verbatim/fancyvrb.pdf
\usepackage{comment}    % for comment environment
%\usepackage{makeidx}    % for \makeindex and \printindex
%\usepackage[english]{babel}      % to use multinational characters
\usepackage{lineno}     % line numbering
\usepackage{afterpage}  % \afterpage{\clearpage}, If it is undesirable to have a pagebreak
   %\afterpage{\clearpage} %% clearpage after the next new page


% For Homework, Quizzes, and Exams
%\usepackage[coverpage,pointsonboth,totalsonright,nosolutions,forpaper]{eqexam}
%\usepackage[nosolutions, forpaper,pointsonboth,totalsonright]{eqexam}
%\usepackage[solutionsafter, forpaper,pointsonboth,totalsonright]{eqexam}




\usepackage[colon]{natbib} % bibliography format   longnamesfirst
        % http://merkel.zoneo.net/Latex/natbib.php
        %   Options that can be added to \usepackage
        %
        %    * round: (default) for round parentheses;
        %    * square: for square brackets;
        %    * curly: for curly braces;
        %    * angle: for angle brackets;
        %    * colon: (default) to separate multiple citations with colons;
        %    * comma: to use commas as separaters;
        %    * authoryear: (default) for author-year citations;
        %    * numbers: for numerical citations;
        %    * super: for superscripted numerical citations, as in Nature;
        %    * sort: orders multiple citations into the sequence in which they appear in the list of references;
        %    * sort&compress: as sort but in addition multiple numerical citations are compressed if possible (as 3-6, 15);
        %    * longnamesfirst: makes the first citation of any reference the equivalent of the starred variant (full author list) and subsequent citations normal (abbreviated list);
        %    * sectionbib: redefines \thebibliography to issue \section* instead of \chapter*; valid only for classes with a \chapter command; to be used with the chapterbib package;
        %    * nonamebreak: keeps all the authors' names in a citation on one line; causes overfull hboxes but helps with some hyperref problems.


%\usepackage{undertilde} % for \utilde environment
\usepackage{totpages}   % total pages
%\usepackage[usenames,dvipsnames]{color}
%\usepackage[dvips]{color}     % color for red box
%\usepackage{fixfoot}  % to repeat a footnote symbol, info 4/7th down in http://edu.stuccess.com/knowcenter/TeX/FAQs/00000008.htm


%%%%%%%%%%%%%%%%%%%%%%%%%%%%%%%%%%%%%%%%%%%%%%%%%%%
%% allow different definitions for using latex and pdflatex
%\newif\ifPDF
%\ifx\pdfoutput\undefined\PDFfalse
%\else\ifnum\pdfoutput > 0\PDFtrue
%       \else\PDFfalse
%       \fi
%\fi
%
%\ifPDF
%%%       %\usepackage[T1]{fontenc}
%%%       %\usepackage{aeguill}
%%%       \usepackage[pdftex]{graphicx}
%%%       \usepackage{color}
%%%       % pdftex/hyperref must be the LAST COMMAND in the preamble
%%%   \IfFileExists{eethesis_slides.tex}{}
%%%   {
%       \usepackage[pdfauthor={Erik Barry Erhardt},%
%            pdfsubject={Dissertation},%
%            pdftitle={Erhardt, Erik Barry, Dissertation: \Title},%
%            pdfkeywords={Statistics, Bayesian},%
%            pdftex]{hyperref}
%       \hypersetup{colorlinks=false,%
%                   citecolor=black,%
%                   filecolor=black,%
%                   linkcolor=black,%
%                   urlcolor=black,%
%                   pdftex}
%%%   } % end %\IfFileExists%
%\else
%%%       %\usepackage[T1]{fontenc}
%%%       \usepackage[dvips]{graphicx}
%%%       %\usepackage[dvips,colorlinks,hyperindex]{hyperref}
%       \usepackage[colorlinks,hyperindex]{hyperref}
%          \def\AnchorColor{anchors}
%          \def\LinkColor{links}
%          \def\pdfBorderAttrs{/Border [0 0 0] } % No border around Links
%\fi


%%% Use:
%%%\begin{theorem}*    (the * will remove numbering if desired)
%%%  ...
%%%\end{theorem}
%%%\begin{proof}[Proof of the Main Theorem]
%%%  ...
%%%\end{proof}


%\newcommand{\bfi}{\bfseries\itshape}



%% start listings
%
   \usepackage{listings}
  \renewcommand{\ttdefault}{pcr}
  \usepackage{bold-extra}  % for bold ttfamily
   %\usepackage[usenames,dvipsnames]{color}
  %\definecolor{Brown}{cmyk}{0,0.81,1,0.60}
  %\definecolor{OliveGreen}{cmyk}{0.64,0,0.95,0.40}
  %\definecolor{CadetBlue}{cmyk}{0.62,0.57,0.23,0}
  %\definecolor{Blue}{rgb}{0,0,1}
  %\definecolor{Purple}{rgb}{1,0,1}
  %\definecolor{DarkGreen}{rgb}{0.0,0.4,0.0}
   \newcommand{\codesize}{\small} %\normalsize}%{\scriptsize}

   \lstloadlanguages{R}%
   \lstset{ language=R,
   %         frame=single,                           % Single frame around code
           frame=ltrb,
           framesep=5pt,
           basicstyle=\codesize\normalfont\ttfamily,
           keywordstyle=\codesize\normalfont\ttfamily,
           %keywordstyle=\normalfont\bfseries\ttfamily,
           %commentstyle=\itshape,
   %        keywordstyle=\color{OliveGreen},
   %        keywordstyle=[1]\color{Blue}\bf,        % MATLAB functions bold and blue
   %        keywordstyle=[2]\color{Purple},         % MATLAB function arguments purple
   %        keywordstyle=[3]\color{Blue}\underbar,  % User functions underlined and blue
   %        %morekeywords=[1]{on, off, interp},
   %        %morekeywords=[2]{on, off, interp},
   %        %morekeywords=[3]{FindESS, homework_example},
   %        identifierstyle=\color{CadetBlue}\bfseries,
   %        commentstyle=\color{Brown},
   %         commentstyle=\usefont{T1}{pcr}{m}{sl}\color{DarkGreen},
   %        stringstyle=\color{DarkGreen}, % Strings
           showstringspaces=false,                 % Don't put marks in string spaces
   %        showstringspaces=true
           tabsize=2                              % 2 spaces per tab
   %        numbers=left,                           % Line numbers on left
   %        firstnumber=1,                          % Line numbers start with line 1
   %        numberstyle=\tiny\color{CadetBlue},     % Line numbers are blue
   %        stepnumber=5                            % Line numbers go in steps of 5
           }
%%   \lstset{language=Matlab,                        % Use MATLAB
%%          frame=single,                           % Single frame around code
%%          basicstyle=\codesize\ttfamily,             % Use small true type font
%%   %       keywordstyle=[1]\color{Blue}\bf,        % MATLAB functions bold and blue
%%   %       keywordstyle=[2]\color{Purple},         % MATLAB function arguments purple
%%   %       keywordstyle=[3]\color{Blue}\underbar,  % User functions underlined and blue
%%          identifierstyle=,                       % Nothing special about identifiers
%%                                                  % Comments small dark green courier
%%   %       commentstyle=\usefont{T1}{pcr}{m}{sl}\color{DarkGreen}\codesize,
%%   %       stringstyle=\color{Purple},             % Strings are purple
%%          showstringspaces=false,                 % Don't put marks in string spaces
%%          tabsize=3,                              % 5 spaces per tab
%%          %
%%          %%% Put standard MATLAB functions not included in the default
%%          %%% language here
%%          morekeywords={xlim,ylim,var,alpha,factorial,poissrnd,normpdf,normcdf},
%%          %
%%          %%% Put MATLAB function parameters here
%%          morekeywords=[2]{on, off, interp},
%%          %
%%          %%% Put user defined functions here
%%          morekeywords=[3]{FindESS, homework_example},
%%          %
%%   %       morecomment=[l][\color{Blue}]{...},     % Line continuation (...) like blue comment
%%          numbers=left,                           % Line numbers on left
%%          firstnumber=1,                          % Line numbers start with line 1
%%   %       numberstyle=\tiny\color{Blue},          % Line numbers are blue
%%          stepnumber=5                            % Line numbers go in steps of 5
%%          }
%%
%%   % Include a C source file. Use like:
%%   %   \csourcefile{stuff.c}{Code to do something or another}
%%   %  First argument is file to include, second is caption for the label
%%   \newcommand{\csourcefile}[2]
%%    {\begin{itemize}\item[]\lstinputlisting[caption=#2,label=#1]{#1}\end{itemize}}
%%   \newcommand{\matlabscript}[2]
%%    {\begin{itemize}\item[]\lstinputlisting[caption=#2,label=#1]{#1}\end{itemize}}
%%   \newcommand{\rsourcefile}[2]
%%    {\begin{itemize}\item[]\lstinputlisting[caption=#2,label=#1]{#1}\end{itemize}}
%%%
%%%% end listings

% end
